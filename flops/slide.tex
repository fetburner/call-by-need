\documentclass[dvipdfmx,cjk,xcolor=dvipsnames,envcountsect,notheorems,12pt]{beamer}
% * 16:9 のスライドを作るときは、aspectratio=169 を documentclass のオプションに追加する
% * 印刷用の配布資料を作るときは handout を documentclass のオプションに追加する
%   (overlay が全て一つのスライドに出力される)
% 図
% 準正規化定理とは何か復習 聴衆が覚えてない
% 標準化定理
% FLOPSに投稿中
% 変更してないのも証明した 書いた方が良い
% 完全性の証明の解説はゆっくり
% 大まかな方針を解説するのでも良い
% スライドを印刷する時は左から右に 隙間に合わせて直す

\usepackage{pxjahyper}% しおりの文字化け対策 (なくても良い)
\usepackage{amsmath,amssymb,amsfonts,amsthm,ascmac,cases,bm,pifont}
\usepackage{graphicx}
\usepackage[thicklines]{cancel}
\usepackage{url}
\usepackage{newtxmath}
\usepackage{bussproofs}
\usepackage{etex}
\usepackage[all]{xy}
\usepackage{mathtools}
\usepackage[absolute, overlay]{textpos}

% スライドのテーマ
\usetheme{sumiilab}
% ベースになる色を指定できる
%\usecolortheme[named=Magenta]{structure}
% 数式の文字が細くて見難い時は serif の代わりに bold にしましょう
%\mathversion{bold}

%% ===============================================
%% スライドの表紙および PDF に表示される情報
%% ===============================================

%% 発表会の名前とか(省略可)
\session{FLOPS 2018}
%% スライドのタイトル
\title{Formal Verification of\\ the Correspondence between \\ Call-by-Need and Call-by-Name}
%% 必要ならば、サブタイトルも
%\subtitle{}
%% 発表者のお名前
\author{\underline{Masayuki Mizuno} and Eijiro Sumii}
%% 発表者の所属([] 内は短い名前)
%\institute[東北大学 住井・松田研]{工学部 電気情報物理工学科\\住井・松田研究室}% 学部生
\institute[Tohoku University Sumii-Matsuda Laboratory]{Tohoku University}% 院生
%% 発表する日
\date{May 9, 2018}

%% ===============================================
%% 自動挿入される目次ページの設定(削除しても可)
%% ===============================================

%% section の先頭に自動挿入される目次ページ(削除すると、表示されなくなる)
\AtBeginSection[]{
\begin{frame}
  \frametitle{Outline}
	\begin{center}
		\begin{minipage}{.9\textwidth}
			\tableofcontents[sectionstyle=show/shaded,subsectionstyle=show/show/hide]
		\end{minipage}
	\end{center}
\end{frame}}
%%% subsection の先頭に自動挿入される目次ページ(削除すると、表示されなくなる)
\AtBeginSubsection[]{
\begin{frame}
  \frametitle{Outline}
	\begin{center}
		\begin{minipage}{.9\textwidth}
			\tableofcontents[sectionstyle=show/shaded,subsectionstyle=show/shaded/hide]
		\end{minipage}
	\end{center}
\end{frame}}
%
%% 現在の section 以外を非表示にする場合は以下のようにする

%% \AtBeginSection[]{
%% \begin{frame}
%%   \frametitle{Outline}
%%   \tableofcontents[sectionstyle=show/hide,subsectionstyle=show/show/hide]
%% \end{frame}}
%% \AtBeginSubsection[]{
%% \begin{frame}
%%   \frametitle{Outline}
%%   \tableofcontents[sectionstyle=show/hide,subsectionstyle=show/shaded/hide]
%% \end{frame}}

%% ===============================================
%% 定理環境の設定
%% ===============================================

%\setbeamertemplate{theorems}[numbered]% 定理環境に番号を付ける
\theoremstyle{definition}
\newtheorem{definition}{Definition}
\newtheorem{axiom}{Axiom}
\newtheorem{theorem}{Theorem}
\newtheorem{lemma}{Lemma}
\newtheorem{corollary}{Corollary}
\newtheorem{proposition}{Proposition}

%% ===============================================
%% ソースコードの設定
%% ===============================================

\usepackage{listings,jlisting}
%\usepackage[scale=0.9]{DejaVuSansMono}

\definecolor{DarkGreen}{rgb}{0,0.5,0}
% プログラミング言語と表示するフォント等の設定
\lstset{
  language={[Objective]Caml},% プログラミング言語
  basicstyle={\ttfamily\small},% ソースコードのテキストのスタイル
  keywordstyle={\bfseries},% 予約語等のキーワードのスタイル
  commentstyle={},% コメントのスタイル
  stringstyle={},% 文字列のスタイル
  frame=trlb,% ソースコードの枠線の設定 (none だと非表示)
  numbers=none,% 行番号の表示 (left だと左に表示)
  numberstyle={},% 行番号のスタイル
  xleftmargin=5pt,% 左余白
  xrightmargin=5pt,% 右余白
  keepspaces=true,% 空白を表示する
  mathescape,% $ で囲った部分を数式として表示する ($ がソースコード中で使えなくなるので注意)
  % 手動強調表示の設定
  moredelim=[is][\itshape]{@/}{/@},
  moredelim=[is][\color{red}]{@r\{}{\}@},
  moredelim=[is][\color{blue}]{@b\{}{\}@},
  moredelim=[is][\color{DarkGreen}]{@g\{}{\}@},
}
\newcommand{\rightarrowdbl}{\rightarrow\mathrel{\mkern-14mu}\rightarrow}

\newcommand{\xtwoheadrightarrow}[2][]{%
  \xrightarrow[#1]{#2}\mathrel{\mkern-14mu}\rightarrow
}

\newcommand{\LET}[3]{\mathbf{let}~#1=#2~\mathbf{in}~#3}
\newcommand{\EXPANDLET}[1]{#1^\pitchfork}
\newcommand{\SIZE}[2]{\parallel #1 \parallel_{#2}}
\newcommand{\FULLBETA}{\xrightarrow{\beta}}
\newcommand{\CALLBYNEED}{\xrightarrow{\mathrm{need}}}
\newcommand{\CALLBYNEEDI}{\xrightarrow{\mathrm{I}}}
\newcommand{\CALLBYNEEDVCA}{\xrightarrow{\mathrm{VCA}}}
\newcommand{\CALLBYNAME}{\xrightarrow{\mathrm{name}}}
\newcommand{\DEMAND}[2]{\mathbf{needs}(#2, #1)}
\newcommand{\STUCK}[2]{\mathbf{needs}_\mathrm{n}(#2, #1)}
\newcommand{\RTCLOS}[1]{#1^*}
\newcommand{\RTCLOSFULLBETA}{\xtwoheadrightarrow{\beta}}
\newcommand{\RTCLOSCALLBYNEED}{\xtwoheadrightarrow{\mathrm{need}}}
\newcommand{\RTCLOSCALLBYNEEDI}{\xtwoheadrightarrow{\mathrm{I}}}
\newcommand{\RTCLOSCALLBYNEEDVCA}{\xtwoheadrightarrow{\mathrm{VCA}}}
\newcommand{\RTCLOSCALLBYNAME}{\xtwoheadrightarrow{\mathrm{name}}}

%% ===============================================
%% 本文
%% ===============================================
\begin{document}
% Thank you for introduction.
% I'd like to talk about ``formal verification of the correspondence between call-by-need and call-by-name''.
\frame[plain]{\titlepage}% タイトルページ

%前半部分をさっさと進む
%倍
\begin{frame}
	% Our motivational problem is a gap between high-level abstraction of non-strict languages---call-by-name evaluation---and actual call-by-need implementations.
	% To bridge this gap, we verify their correspondence.
	\frametitle{Motivation: a gap between abstraction and implementations of non-strict languages}
	\begin{itemize}
		\item Call-by-name [Abramsky 1990 etc.]:\\(high-level) abstraction of non-strict languages
		\item Call-by-need [Wadsworth 1971 etc.]:\\implementations of non-strict languages
	\end{itemize}
	\vfill
	\begin{center}
		\Large
		Our goal is mechanized verification of\\ their correspondence
	\end{center}
\end{frame}

\begin{frame}
	% Full-beta reduction is one of a theoretical background of functional languages.
	% But it's non-deterministic.
	% Inconsiderate reduction may causes infinite loop.
	\frametitle{Background 1: full-$\beta$ reduction}
	\vspace{-\baselineskip}
	\begin{itemize}
		\item Reduction is non-deterministic
			\begin{itemize}
				\item $\underline{(\lambda xy.~y)~\Omega} \FULLBETA \lambda y.~y$
			\end{itemize}
			\begin{itemize}
				\item $(\lambda xy.~y)~\underline{\Omega} \FULLBETA (\lambda xy.~y)~\Omega$
			\end{itemize}
	\end{itemize}
	\vfill
	\begin{flushright}
		\Large $\Omega=(\lambda x.~xx)~(\lambda x.~xx)$
	\end{flushright}
\end{frame}

\begin{frame}
	% Call-by-name is a better abstraction of functional languages.
	% It reduces the head of applications, so deterministic.
	% In addition, terms which are evaluable by full-beta reduction, are also evaluable by call-by-name reduction.
	\frametitle{Background 2: call-by-name}
	\begin{definition}[call-by-name]
		\Large
		\[ E_\mathrm{n} \vcentcolon\vcentcolon= [] ~|~ E_\mathrm{n}~M \]
		\[ E_\mathrm{n}[(\lambda x.M)N] \CALLBYNAME E_\mathrm{n}[M[x \mapsto N]] \]
	\end{definition}
	\begin{itemize}
		\item If $M \RTCLOSFULLBETA \lambda x. N$, then $M \RTCLOSCALLBYNAME \lambda x. N'$
	\end{itemize}
	\Large \[(\lambda xy.~y)~\Omega \RTCLOSCALLBYNAME \lambda y.~y\]
\end{frame}

\begin{frame}
	% However, call-by-name sometimes needs redundant reductions due to redex duplication.
	% For example, let me evaluate a term (\lambda x.~xx)~(I~I).
	% Although full-beta reduction can evaluate in 3 steps, call-by-name reduction needs 4 steps!
	\frametitle{Problem: Redundant reductions}
	\Large
	\begin{columns}[t]
		\begin{column}{0.3\textwidth}
			\[\begin{array}{l}
				(\lambda x.~xx)~\underline{(I~I)} \\
				\FULLBETA \underline{(\lambda x.~xx)~I} \\
				\FULLBETA \underline{I~I} \\
				\FULLBETA I \\
			\end{array}\]
		\end{column}
		\begin{column}{0.4\textwidth}
			\[\begin{array}{l}
				\underline{(\lambda x.~xx)~(I~I)} \\
				\CALLBYNAME \underline{I~I}~(I~I) \\
				\CALLBYNAME \underline{I~(I~I)} \\
				\CALLBYNAME \underline{I~I} \\
				\CALLBYNAME I \\
			\end{array}\]
		\end{column}
	\end{columns}
	\vfill
	\begin{flushright}
		\Large $I=\lambda x.~x$
	\end{flushright}
\end{frame}

\begin{frame}
	% Therefore, actual implementations reuses the result of evaluation, namely, call-by-need reduction.
	% Its behaviour should correspond with call-by-name.
	\frametitle{Background 3: call-by-need}
	\begin{itemize}
		\item Reuse evaluation
			\[\begin{array}{l}
				(\lambda x.~xx)~(I~I) \\
				\CALLBYNEED \LET{x}{\underline{I~I}}{x~x} \\
				\CALLBYNEED \underline{\LET{x}{I}{x~x}} \\
				\CALLBYNEED \LET{x}{I}{\underline{I~x}} \\
			\end{array}\]
		\item Should correspond with call-by-name
	\end{itemize}
\end{frame}

\begin{frame}
	% Our contribution are twofold;
	% formalization of Ariola and felleisen's call-by-need lambda calculus in Coq proof assistant and
	% a simplified proof of correspondence with call-by-name, and verification in Coq.
	% Our proof uses Curry and Feys's standardization theorem.
	% In this presentation, I'd like to focus on this simplified proof.
	\frametitle{Our contributions}
	\begin{itemize}
		\item Formalization of call-by-need $\lambda$-calculus \mbox{[Ariola+ 1995]} in the Coq proof assistant
		\vfill
		\item \alert{Simplified proof} of correspondence with call-by-name, and verification in Coq
			\begin{itemize}
				\item using standardization theorem \mbox{[Curry\&Feys 1958]}
			\end{itemize}
	\end{itemize}
\end{frame}

% The outline of this presentation is as follows.
\section*{Outline}
\begin{frame}
  \frametitle{Outline}
	\begin{center}
		\begin{minipage}{.9\textwidth}
			\tableofcontents[sectionstyle=show,subsectionstyle=hide]
		\end{minipage}
	\end{center}
\end{frame}

% First, I'd like to explain the target languages: call-by-name $\lambda$-calculus and call-by-need $\lambda$-calculus
\section{Call-by-name and call-by-need $\lambda$-calculi}

\begin{frame}
	% The definition of first target language, call-by-name lambda calculus is as follows.
	% Terms are the same with ordinal lambda calculus.
	% Values are lambda abstractions.
	% Evaluation contexs are either hole or the left side of application.
	% In this language, reduction is deterministic and all stuck states are represented by evaluation contexts and variables
	\frametitle{Call-by-name $\lambda$-calculus}
	\large
	\vspace{-5mm}
	\[ \begin{array}{llcl}
		\mbox{Terms} & L,M,N & \vcentcolon\vcentcolon= & x~|~V~|~M~N \\
		\mbox{Values (WHNF)} & V & \vcentcolon\vcentcolon= & \lambda x.M \\
		\mbox{Evaluation contexts} & E_\mathrm{n} & \vcentcolon\vcentcolon= & [] ~|~ E_\mathrm{n}~M \\
	\end{array} \]
	\vfill
	\[\frac{M \rightarrow N}{E_\mathrm{n}[M] \CALLBYNAME E_\mathrm{n}[N]}\]
	\vfill
	\[ \begin{array}{lrcl}
		(\beta)\quad & (\lambda x.M)N & \rightarrow & M[x \mapsto N] \\
	\end{array} \]
	\vspace{-5mm}
	\begin{itemize}
		\item Reduction is deterministic
		\item All stuck states are of the form $E_\mathrm{n}[x]$
	\end{itemize}
\end{frame}

\begin{frame}
	% These properties are stated as follows.
	% Call-by-name reduction is a partial function.
	% Contents of stuck states are unique.
	% For any term $M$, exactly one of the following holds:
	% $M$ is a value,
	% $M$ is a stuck state or
	% $M$ is reducible by call-by-name reduction.
	% That is, no terms are value and also stuck state.
	\begin{lemma}[determinacy of call-by-name reduction]
	\begin{itemize}
		\item $\CALLBYNAME$ is partial function
		\item If $E_\mathrm{n}[x]=E'_\mathrm{n}[y]$ then $x=y$
		\item For any term $M$, \alert{exactly one} of the following holds:
			\begin{enumerate}
				\item $M$ is a value
				\item $M=E_\mathrm{n}[x]$ for some $E_\mathrm{n}$ and $x$
				\item $M \CALLBYNAME N$ for some $N$
			\end{enumerate}
	\end{itemize}
	\end{lemma}
\end{frame}

\begin{frame}
	% Now, I'd like to introduce a important property, standardization theorem.
	% It means that any reduction sequences are sortable by reduction position in left-to-right, outside-to-inside order.
	\frametitle{Standardization theorem [Curry\&Feys 1958]}
	\Large
	\begin{definition}[standard reduction sequence]
		A reduction sequence $M_1\xrightarrow[\Delta_1]{\beta}M_2\xrightarrow[\Delta_2]{\beta} \cdots \xrightarrow[\Delta_{n-1}]{\beta} M_n$
		is \textit{standard} if every $\Delta_i$ is outer and lefter than $\Delta_{i+1}$
	\end{definition}
	\begin{theorem}[standardization]
		If $M\RTCLOSFULLBETA N$, then there is a standard reduction sequence from $M$ to $N$
	\end{theorem}
\end{frame}

\begin{frame}
	% Standardization theorem gives two important corollaries.
	% First, the termination of call-by-name reduction:
	% terms which are evaluable by full-beta reduction, are also evaluable by call-by-name reduction.
	% Second, the termination of the composition of call-by-name reduction and multi-step full-beta reductions:
	% If M is evaluable, any reductions beginning at M always terminate despite non-determinacy.
	% These corollaries are used for our proof of the correspondence with call-by-need.
	\frametitle{Corollaries}
	\Large
	\begin{corollary}[termination of $\CALLBYNAME$]
		If $M \RTCLOSFULLBETA V$ then, $M \RTCLOSCALLBYNAME V'$ for some $V'$
	\end{corollary}
	\begin{corollary}[termination of ${\CALLBYNAME}\circ{\RTCLOSFULLBETA}$]
		If $M \RTCLOSFULLBETA V$, then
		$M$ is terminating by ${\CALLBYNAME}\circ{\alert{\RTCLOSFULLBETA}}$
	\end{corollary}
	\begin{itemize}
		\item Used for our proof of the correpondence with call-by-need
	\end{itemize}
\end{frame}

\begin{frame}
	\frametitle{Call-by-need $\lambda$-calculus [Ariola+ 1995]}
	% The definition of second target language, Ariola and Felleisen's call-by-need lambda calculus is as follows.
	% A syntactic form let x = M in N is introduced for representing M contained thunk is referenced as x by N.
	% The reduction rule (I) corresponding with function calls create thunk instead of substituting.
	% The content of a thunk is reduced when needed.
	% And be substituted if evaluated.
	% Now, we define reductions made from specific rules.
	% Arrow I is reduction only using rule (I).
	% This reduction corresponds actual reduction.
	% Arrow VCA is reduction only using rule (I), (C) and (A).
	% This reduction is administrative.
	{\large
		\vspace{-1cm}
		\[\begin{array}{llcl}
			\mbox{Terms} & M,N & \vcentcolon\vcentcolon = & x~|~V~|~M~N~|~\LET{x}{M}{N} \\
			\mbox{Values} & V & \vcentcolon\vcentcolon = & \lambda x.~M \\
			\mbox{Answers} & A & \vcentcolon\vcentcolon = & V~|~\LET{x}{M}{A} \\
			\mbox{Evalctx} & E,E' & \vcentcolon\vcentcolon = & [] ~|~ E~M ~|~\LET{x}{M}{E} \\
			& & | & \LET{x}{E}{E'[x]} \\
		\end{array}\] }
	\vspace{-3mm}
	{\large
		\[ \begin{array}{ll}
			\alert{(I)} & \alert{(\lambda x.M)N} \alert{\rightarrow} \alert{\LET{x}{N}{M}} \\
			(V) & \LET{x}{V}{E[x]} \rightarrow \LET{x}{V}{E[V]} \\
			(C) & (\LET{x}{M}{A})~N \rightarrow \LET{x}{M}{A~N} \\
			(A) & \LET{y}{(\LET{x}{M}{A})}{E[y]} \\
					& \rightarrow \LET{x}{M}{\LET{y}{A}{E[y]}} \\
		\end{array}\]}
	\vspace{-3mm}
	{\normalsize
		\[ \begin{array}{ll}
			\CALLBYNEEDI & \mbox{\alert{reduction only using (I)}}\\
			\CALLBYNEEDVCA & \mbox{reduction only using (V), (C) and (A) (administrative)}
		\end{array} \]}
\end{frame}

\begin{frame}
	% Call-by-need reduction is also deterministic.
	\begin{lemma}[determinacy of call-by-need reduction]
	\begin{itemize}
		\item $\CALLBYNEEDI$ is a partial function
		\item $\CALLBYNEEDVCA$ is a partial function
		\item If $E[x]=E'[y]$ then $x=y$
		\item For any term $M$, \alert{exactly one} of the following holds:
			\begin{enumerate}
				\item $M$ is an answer
				\item $M=E[x]$ for some $E$ and $x$
				\item $M\CALLBYNEEDI N$ for some $N$
				\item $M\CALLBYNEEDVCA N$ for some $N$
			\end{enumerate}
	\end{itemize}
	\end{lemma}
\end{frame}

% Next, I'd like to talk about the subject of this presentation, simplified proof of the correspondence.
\section{Simplified proof of the correspondence}

\begin{frame}[fragile]
	% Let us define the correspondence of terms by let expansion.
	% It's almost equivalent to Maraist et alii's definition expect marked redexes.
	\Large
	\begin{definition}[correspondence of terms]
		$\pitchfork : \mbox{call-by-need terms} \rightarrow \mbox{call-by-name terms}$ \\
	$\begin{array}{rcl}
		\EXPANDLET{x} & = & x \\
		\EXPANDLET{(\lambda x. M)} & = & \lambda x. \EXPANDLET{M} \\
		\EXPANDLET{(M~N)} & = & \EXPANDLET{M}~\EXPANDLET{N} \\
		\EXPANDLET{(\LET{x}{M}{N})} & = & \EXPANDLET{N}[x \mapsto \EXPANDLET{M}] \\
	\end{array}$
	\end{definition}
	\begin{itemize}
		\item Expands $\mathbf{let}$
		\item Equivalent to [Maraist+ 1998] \\(except ``marked redexes'')
	\end{itemize}
\end{frame}

\begin{frame}
	% Now, the main theorem, correspondence of call-by-need with call-by-name is stated as follows.
	% Forward direction, soundness of call-by-need is
	% ``If m is evaluated to an answer, M pitchfork is evaluable''.
	% Backward direction, completeness of call-by-need is
	% ``If M pitchfork is evaluated, M is evaluable to an answer''.
	% Note that the correspondence between the answer and the values also holds.
	\frametitle{Main theorem: correspondence of call-by-need with call-by-name}
	\Large
	\begin{theorem}[soundness of $\CALLBYNEED$]
		If $M\RTCLOSCALLBYNEED A$, then $\EXPANDLET{M}\RTCLOSCALLBYNAME V$ for some $V$
	\end{theorem}
	\begin{theorem}[completeness of $\CALLBYNEED$]
		If $\EXPANDLET{M}\RTCLOSCALLBYNAME V$, then $M\RTCLOSCALLBYNEED A$ for some $A$
	\end{theorem}
	\begin{center}
		\Large (Correspondence between $A$ and $V$ also holds)
	\end{center}
\end{frame}

\begin{frame}[fragile]
	% Let us reference previous researches
	% Ariola and Felleisen's proof is not suitable for formal verification because it's based on informally defined term graphs and their correspondence.
	% Maraist et alii's proof is convenient for neither mechanization nor paper-and-pencil due to complicated ``marked reduction'' and explicit treatment of reduction position
	% So we gave a simpler proof!
	\frametitle{Cf. previous researches}
	\begin{itemize}
		\item Ariola and Felleisen [1997]
			\begin{itemize}
				\item Based on informally defined term graphs and their correspondence
			\end{itemize}
	\end{itemize}
	\begin{itemize}
		\item Maraist et al. [1998]
			\begin{itemize}
				\item Complicated ``marked reduction'' and explicit treatment of reduction position
			\end{itemize}
	\end{itemize}
	\begin{center}
		\begin{minipage}{.67\linewidth}
			\begin{block}{}
				\LARGE
				We give a \alert{simpler proof}!
			\end{block}
		\end{minipage}
	\end{center}
\end{frame}

\begin{frame}
	% I'd like to explain our proof.
	% For preparation, we state the lemmma, single-step correspondence.
	% Answers correspond value.
	% Stuck states in call-by-need correspond stuck states in call-by-name.
	% Arrow VCA is administrative.
	% Arrow I corresponds the composition of call-by-name reduction and additional full-beta reductions.
	\frametitle{Our proof}
	\begin{lemma}[single-step correspondence]
	\begin{itemize}
		\item $\EXPANDLET{A}$ is a value
		\item For any $E$ and $x$, there exists $E_\mathrm{n}$ such that $\EXPANDLET{E[x]} = E_\mathrm{n}[x]$
		\item If $M \CALLBYNEEDVCA N$ then \alert{$\EXPANDLET{M} = \EXPANDLET{N}$}
		\item If $M \CALLBYNEEDI N$ then $\EXPANDLET{M}{\CALLBYNAME}\circ{\alert{\RTCLOSFULLBETA}} \EXPANDLET{N}$
	\end{itemize}
	\end{lemma}
\end{frame}

\begin{frame}
	% The forward direction of main theorem, soundness of call-by-need is easily proved.
	% By assumption, M pitchfork is reduced to A pitchfork by multi-step full-beta reduction
	% Since A is an answer, A pitchfork is a value
	% Hence M pitchfork is evaluated by the termination of call-by-name reduction
	\frametitle{Proof of soundness}
	\begin{proof}[]
		\Large
		\[\xymatrix{
			M \ar@{->>}[rrrr]^-{\mathrm{need}} \ar@{.>}[d]_-\pitchfork & & & & A \ar@{.>}[d]_-\pitchfork \\
			\EXPANDLET{M} \ar@{->>}[rrrr]_-{\beta} \ar@{->>}@/_10pt/@[red][rrrd]_-{\alert{\mathrm{name}}} & & & & \EXPANDLET{A} \\
			& & & \exists V &
		} \]

		Since $A$ is an answer, $\EXPANDLET{A}$ is a value \\
		Hence $\EXPANDLET{M}\RTCLOSCALLBYNAME V$ by the termination of $\CALLBYNAME$
	\end{proof}
\end{frame}

\begin{frame}
	\frametitle{Completeness}
	% The backward direction of main theorem, completeness of call-by-need is harder than forward direction.
	% Let us review the statement.
	% If M pitchfork evaluated, then M evaluable to a answer.
	% The difficulties are twofold.
	% First, administrative reduction might not terminate.
	% Second, arrow I corresponds the composition of call-by-need reduction and multi-step full-beta reduction.
	% That is bodies of lambda abstraction may be reduced.
	\Large
	\vspace{-7mm}
	\begin{center}
		\begin{minipage}{.65\linewidth}
			\begin{block}{}
				If $\EXPANDLET{M}\RTCLOSCALLBYNAME V$ then $M\RTCLOSCALLBYNEED A$ \\
			\end{block}
		\end{minipage}
	\end{center}
	\vfill
	Difficulties:
	\begin{itemize}
		\item Administrative reductions might not terminate
			\begin{itemize}
				\item If $M\CALLBYNEEDVCA N$ then $\EXPANDLET{M}=\EXPANDLET{N}$
			\end{itemize}
		\item Redexes shared by \textbf{let} are reduced at once
			\begin{itemize}
				\item $M\CALLBYNEEDI N$ implies $\EXPANDLET{M}{\CALLBYNAME}\circ{\alert{\RTCLOSFULLBETA}}\EXPANDLET{N}$
					\begin{itemize}
						\item Bodies of $\lambda$-abstraction can be reduced
					\end{itemize}
			\end{itemize}
	\end{itemize}
\end{frame}

\begin{frame}
	% To solve the first difficulty, we show the normalization of administrative reduction.
	% This lemma is proved by a variant of Maraist et alii's weighting.
	\Large
	\begin{lemma}[normalization of $\CALLBYNEEDVCA$]
		By a variant of {[Maraist+ 1998]}'s weighting:
		{\normalsize \[ \begin{array}{rcl}
		\SIZE{x}{s} & = & s(x) \\
		\SIZE{\lambda x. M}{s} & = & \SIZE{M}{s \circ [x \mapsto 1]} \\
		\SIZE{M~N}{s} & = & 2\SIZE{M}{s}+2\SIZE{N}{s} \\
		\SIZE{\LET{x}{M}{N}}{s} & = & 2\SIZE{M}{s}+\SIZE{N}{s \circ [x \mapsto 1+\SIZE{M}{s}]}
		\end{array} \]}
		$M\CALLBYNEEDVCA N$ implies $\SIZE{M}{s}>\SIZE{N}{s}$
	\end{lemma}
\end{frame}

\begin{frame}
	% Now, I'd like to show the completeness of call-by-need.
	% First, we show the call-by-need reduction of M is normalizing.
	% Since call-by-need reduction corresponds the union of arrow I and arrow VCA, multi-step call-by-need reduction corresponds the repetition of the composition of multi-step arrow VCA and arrow I.
	% Note that arrow VCA is administrative reduction and arrow I corresponds the composition of call-by-name reduction and multi-step full-beta reduction.
	% Because M pitchfork is evaluable by full-beta reduction, M pitchfork is terminating by the composition of call-by-name reduction and multi-step full-beta reduction.
	% Namely, induction on derivation is available.
	\large
	\begin{textblock*}{\linewidth}(1em, 0pt)
		\begin{proof}[Proof (completeness of call-by-need) (1/2)]
			Assume $\EXPANDLET{M}\RTCLOSCALLBYNAME V$, show $M\RTCLOSCALLBYNEED A$
			\begin{center}
				\begin{minipage}{.95\linewidth}
					\begin{beamercolorbox}[shadow=true, rounded=true]{frametitle}
						First, we show call-by-need reduction of $M$ is normalizing
					\end{beamercolorbox}
				\end{minipage}
			\end{center}
			\vspace{-5mm}
			\only<2-4>{
				\begin{center}
					\begin{minipage}{.95\linewidth}
						\begin{beamercolorbox}[shadow=true, rounded=true]{frametitle}
							\only<2>{${\RTCLOSCALLBYNEED} = \RTCLOS{({\CALLBYNEEDI}\cup{\CALLBYNEEDVCA})} = \RTCLOS{({\RTCLOSCALLBYNEEDVCA}\circ\CALLBYNEEDI)}\circ{\RTCLOSCALLBYNEEDVCA}$ holds}
							\only<3>{
								\hspace{-5mm}
								${\CALLBYNEEDVCA}$ is an administrative reduction

								${\CALLBYNEEDI}$ corresponds ${\CALLBYNAME}\circ{\RTCLOSFULLBETA}$}
							\only<4>{
								\hspace{-6mm}
								By $\EXPANDLET{M} \RTCLOSFULLBETA V$, $\EXPANDLET{M}$ is terminating by ${\CALLBYNAME}\circ{\RTCLOSFULLBETA}$ (= induction on derivation is available)}
						\end{beamercolorbox}
					\end{minipage}
				\end{center}
				\vspace{-5mm}}
			\phantom\qedhere
		\end{proof}
	\end{textblock*}
	\begin{textblock*}{0.6\linewidth}(80pt, 190pt)
		\centering
		\only<1>{
			$\xymatrix{
				M \ar@{->}[rr]^-{\mathrm{need}} & & \ar@{--}[r] & \ar@{->}[rr]^-{\mathrm{need}} & & \\
				&
			}$}
		\only<2>{
			$\xymatrix{
				M \ar@{->>}@[red][r]^-{\alert{\mathrm{VCA}}} & \ar@[red][r]^-{\alert{\mathrm{I}}} & \ar@[red]@{--}[r] & \ar@[red]@{->>}[r]^-{\alert{\mathrm{VCA}}} & \ar@[red][r]^-{\alert{\mathrm{I}}} & \ar@[red]@{->>}[r]^-{\alert{\mathrm{VCA}}} & \\
				&
			}$}
		\only<3>{
			$\xymatrix{
				M \ar@{->>}@[red][r]^-{\alert{\mathrm{VCA}}} \ar@{.>}[d]_-\pitchfork & \ar@[red][r]^-{\alert{\mathrm{I}}} \ar@/_10pt/@{.>}[ld]^-\pitchfork & \ar@{.>}[d]^-\pitchfork \ar@[red]@{--}[r] & \ar@[red]@{->>}[r]^-{\alert{\mathrm{VCA}}} \ar@{.>}[d]_-\pitchfork & \ar@[red][r]^-{\alert{\mathrm{I}}} \ar@/_10pt/@{.>}[ld]^-\pitchfork & \ar@[red]@{->>}[r]^-{\alert{\mathrm{VCA}}} \ar@{.>}[d]_-\pitchfork & N \ar@/_10pt/@{.>}[ld]^-\pitchfork\\
				\EXPANDLET{M} \ar[r]^-{\mathrm{name}} & \ar@{->>}[r]^-{\beta} & \ar@{--}[r] & \ar[r]^-{\mathrm{name}} & \ar@{->>}[r]^-{\beta} & \EXPANDLET{N} \\
			}$}
		\only<4>{
			$\xymatrix{
				M \ar@{->>}@[red][r]^-{\alert{\mathrm{VCA}}} \ar@{.>}[d]_-\pitchfork & \ar@[red][r]^-{\alert{\mathrm{I}}} \ar@/_10pt/@{.>}[ld]^-\pitchfork & \ar@{.>}[d]^-\pitchfork \ar@[red]@{--}[r] & \ar@[red]@{->>}[r]^-{\alert{\mathrm{VCA}}} \ar@{.>}[d]_-\pitchfork & \ar@[red][r]^-{\alert{\mathrm{I}}} \ar@/_10pt/@{.>}[ld]^-\pitchfork & \ar@[red]@{->>}[r]^-{\alert{\mathrm{VCA}}} \ar@{.>}[d]_-\pitchfork & N/ \ar@/_10pt/@{.>}[ld]^-\pitchfork\\
				\EXPANDLET{M} \ar[r]^-{\mathrm{name}} & \ar@{->>}[r]^-{\beta} & \ar@{--}[r] & \ar[r]^-{\mathrm{name}} & \ar@{->>}[r]^-{\beta} & \EXPANDLET{N}/ \\
			}$}
	\end{textblock*}
\end{frame}

\begin{frame}
	% Next, we show the normal form N of M is an answer.
	% The normal form N is an answer of stuck state.
	% Assume N is stuck state, show it leads to contradiction.
	% By singlestep correspondence, M pitchfork is reduced into N pitchfork
	% E[x] pitchfork is also stuck state E_n[x]
	% By confluence of full-beta reduction, V and E_n[x] is conflate into L.
	% Full-beta reduction preserves valueness and stuckness in call-by-name.
	% Namely, L is a value and also stuck state.
	% It leads to contradiction.
	\large
	\begin{textblock*}{\linewidth}(1em, 0pt)
		\begin{proof}[Proof (completeness of call-by-need) (2/2)]
			\begin{center}
				\begin{minipage}{.95\linewidth}
					\begin{beamercolorbox}[shadow=true, rounded=true]{frametitle}
						Next, show normal form $N$ of $M$ is an answer
					\end{beamercolorbox}
				\end{minipage}
			\end{center}
			\only<2->{
				\vspace{-5mm}
				\begin{center}
					\begin{minipage}{.95\linewidth}
						\begin{beamercolorbox}[shadow=true, rounded=true]{frametitle}
							\only<2>{Normal form $N$ is an answer or stuck state $E[x]$}
							\only<3>{\hspace{-1.5mm}Assume $N$ is stuck state, show it leads to contradiction}
							\only<4-5>{\hspace{-3mm}By single-step correspondence:}
							\only<6>{\hspace{-4.5mm}By assumption:}
							\only<7>{\hspace{-6mm}By confluence of $\FULLBETA$:}
							\only<8>{\hspace{-7.5mm}$\FULLBETA$ preserves valueness and stuckness in call-by-name}
						\end{beamercolorbox}
					\end{minipage}
				\end{center}}
			\only<1-7>{\vspace{-5mm}\phantom\qedhere}
			\only<8>{\qedhere}
		\end{proof}
	\end{textblock*}
	\begin{textblock*}{0.6\linewidth}(50pt, 150pt)
		\centering
		\only<1>{
			$\xymatrix{
				& & M \ar@{->>}[rd]^-{\mathrm{need}}\\
				& & & N \\
				&
			}$}
		\only<2>{
			$\xymatrix{
				& & M \ar@{->>}[rd]^-{\mathrm{need}}\\
				& & & N=A\vee N=E[x]  \\
				&
			}$}
		\only<3>{
			$\xymatrix{
				& & M \ar@{->>}[rd]^-{\mathrm{need}}\\
				& & & N=E[x]  \\
				&
			}$}
		\only<4>{
			$\xymatrix{
				& \EXPANDLET{M} \ar@{->>}[rd]^-\beta & M \ar@{.>}[l]_-{\pitchfork} \ar@{->>}[rd]^-{\mathrm{need}}\\
				& & \EXPANDLET{(E[x])} & N=E[x] \ar@{.>}[l]^-{\pitchfork} \\
				&
			}$}
		\only<5>{
			$\xymatrix{
				& \EXPANDLET{M} \ar@{->>}[rd]^-\beta & M \ar@{.>}[l]_-{\pitchfork} \ar@{->>}[rd]^-{\mathrm{need}}\\
				& & \alert{E_\mathrm{n}[x]} & N=E[x] \ar@{.>}[l]^-{\pitchfork} \\
				&
			}$}
		\only<6>{
			$\xymatrix{
				& \EXPANDLET{M} \ar@{->>}[rd]^-\beta \ar@{->>}[ld]_-{\mathrm{name}} & M \ar@{.>}[l]_-{\pitchfork} \ar@{->>}[rd]^-{\mathrm{need}}\\
				V & & E_\mathrm{n}[x] & N=E[x] \ar@{.>}[l]^-{\pitchfork} \\
				&
			}$}
		\only<7>{
			$\xymatrix{
				& \EXPANDLET{M} \ar@{->>}[rd]^-\beta \ar@{->>}[ld]_-{\mathrm{name}} & M \ar@{.>}[l]_-{\pitchfork} \ar@{->>}[rd]^-{\mathrm{need}}\\
				V \ar@{-->>}[rd]_-\beta & & E_\mathrm{n}[x] \ar@{-->>}[ld]^-\beta & N=E[x] \ar@{.>}[l]^-{\pitchfork} \\
				& L
			}$}
		\only<8>{
			$\xymatrix{
				& \EXPANDLET{M} \ar@{->>}[ld]_-{\mathrm{name}} \ar@{->>}[rd]^-\beta & M \ar@{.>}[l]_-{\pitchfork} \ar@{->>}[rd]^-{\mathrm{need}}\\
				V \ar@{-->>}[rd]_-\beta & & E_\mathrm{n}[x] \ar@{-->>}[ld]^-\beta & N=E[x] \ar@{.>}[l]^-{\pitchfork} \\
				& L=V'=E_\mathrm{n}'[x] \\
			}$}
	\end{textblock*}
\end{frame}

% Next, I'd like to talk about Coq formalization.
\section{Coq formalization}

\begin{frame}[fragile]
	% Our formalization is almost straightforward, expect treatment of evaluation contexts.
	% Let us show the determinacy of call-by-need reduction by induction on M.
	\frametitle{Coq formalization}
	\Large Almost straightforward, expect treatment of evaluation contexts
\begin{lstlisting}
Lemma answer_or_stuck_or_reducible M :
  answer M
  ∨ (∃ E x,
    evalctx E ∧ M = E.[tvar x] ∧ bv E ≦ x)
  ∨ (∃ E L N,
    evalctx E ∧ M = E.[L] ∧ reduceI L N)
  ∨ (∃ E L N,
    evalctx E ∧ M = E.[L] ∧ reduceVCA L N).
\end{lstlisting}
	\begin{itemize}
		\item Try induction on $M$
	\end{itemize}
\end{frame}

\begin{frame}[fragile]
	% The first case is variable.
	% It is trivial from E equals hole.
	\frametitle{Case $M=x$}
\begin{lstlisting}
x : var
============================
answer (tvar x)
∨ (∃ E y, evalctx E ∧ tvar x = E.[tvar y]
  ∧ bv E ≦ x)
∨ (∃ E L N, evalctx E ∧ tvar x = E.[L]
  ∧ reduceI L N)
∨ (∃ E L N, evalctx E ∧ tvar x = E.[L]
  ∧ reduceVCA L N).
\end{lstlisting}
	\begin{itemize}
		\item Trivial from $E = []$
	\end{itemize}
\end{frame}

\begin{frame}[fragile]
	% However, automated reasoning fails.
	\frametitle{Case $M=x$}
	\begin{itemize}
		\item However, automated reasoning fails
	\end{itemize}
\begin{lstlisting}
Coq < eauto.

x : var
============================
answer (tvar x)
∨ (∃ E y, evalctx E ∧ tvar x = E.[tvar y]
  ∧ bv E ≦ x)
∨ (∃ E L N, evalctx E ∧ tvar x = E.[L]
  ∧ reduceI L N)
∨ (∃ E L N, evalctx E ∧ tvar x = E.[L]
  ∧ reduceVCA L N).
\end{lstlisting}
\end{frame}

\begin{frame}[fragile]
	% Why reasoning fails?
	% To prove the subgoal, we must find E such that x = E.[y]
	% Since hole filling is implemented as function, higher order pattern matching is needed!
	\frametitle{Why fails?}
\Large
To prove...
\begin{lstlisting}
∃ E y, evalctx E ∧
  tvar x = E.[tvar y] ∧ bv E ≦ x
\end{lstlisting}
...we must find $E$ such that
\begin{lstlisting}
tvar x = E.[tvar y]
\end{lstlisting}
	\vfill
	\begin{center}{\fontsize{30pt}{25pt}\selectfont $\Downarrow$}\end{center} % XXX 雑
	\vfill
	\begin{center}
		Higher order pattern matching required!
	\end{center}
	\vfill
\end{frame}

\begin{frame}
	% Our solution is the elimination of evaluation contexts.
	% we expand evaluation contexts into reductions,
	% introduce stuckness predicate,
	% and approximate the reduction rule (V) by substitution.
	% Note that the correspondence in original semantics is also proved.
	\frametitle{Solution: eliminate evaluation contexts}
	\vspace{-5mm}
	\begin{itemize}
		\item Expand evaluation contexts into reductions
			\vspace{-5mm}
			\begin{itemize}
				\item $\FULLBETA$, $\CALLBYNAME$, $\CALLBYNEEDI$ and $\CALLBYNEEDVCA$
			\end{itemize}
		\item Introduce stuckness predicate
			\begin{itemize}
				\item $\STUCK{x}{M}~(\Leftrightarrow \exists E. M=E_\mathrm{n}[x])$ and $\DEMAND{x}{M}~(\Leftrightarrow \exists E. M=E[x])$
			\end{itemize}
		\item Approximate\\
			${\LET{x}{V}{E[x]}}\rightarrow{\LET{x}{V}{E[V]}}$\\
			by substitution\\
			${\LET{x}{V}{E[x]}}\rightarrow{E[x][x\mapsto V]}$\\
	\end{itemize}
	\begin{flushright}
		(N.B. correspondence in original semantics is also proved)
	\end{flushright}
\end{frame}

\begin{frame}[fragile]
	% Thanks to the elimination of evaluation contexts, proof automation succeeds as follows
	\frametitle{Automation succeeds!}
\begin{lstlisting}
Lemma answer_or_stuck_or_reducible M :
  answer M \/
  (exists x, needs M x) \/
  (exists N, reduceI M N) \/
  (exists N, reduceVCA M N).
Proof.
  induction M as
    [|? [Hanswer|[[]|[[]|[]]]]
    ||? [Hanswer|[[]|[[]|[]]]]
      ? [|[[[]]|[[]|[]]]]]; eauto 6;
    inversion Hanswer; subst; eauto 6.
Qed.
\end{lstlisting}
\end{frame}

% Finally, I'd like to conclude.
\section{Conclusion}

\begin{frame}
	\frametitle{Conclusion}
	% We formalized Ariola and felleisen's call-by-need lambda calculus in Coq proof assistant,
	% gave standardization theorem based simplified proof of correspondence of call-by-name, and verified in Coq.
	% Thank you for listening.
	\begin{itemize}
		\item Formalized call-by-need $\lambda$-calculus \mbox{[Ariola+ 1995]} in the Coq proof assistant
		\vfill
		\item Gave \alert{simplified proof} of correspondence with call-by-name, and verified in Coq
			\begin{itemize}
				\item Using standardization theorem [Curry\&Feys 1958]
			\end{itemize}
	\end{itemize}
\end{frame}

\end{document}

