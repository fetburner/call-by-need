\documentclass[dvipdfmx,cjk,xcolor=dvipsnames,envcountsect,notheorems,12pt]{beamer}
% * 16:9 のスライドを作るときは、aspectratio=169 を documentclass のオプションに追加する
% * 印刷用の配布資料を作るときは handout を documentclass のオプションに追加する
%   (overlay が全て一つのスライドに出力される)

\usepackage{pxjahyper}% しおりの文字化け対策 (なくても良い)
\usepackage{amsmath,amssymb,amsfonts,amsthm,ascmac,cases,bm,pifont}
\usepackage{graphicx}
\usepackage[thicklines]{cancel}
\usepackage{url}
\usepackage{newtxmath}
\usepackage{bussproofs}

% スライドのテーマ
\usetheme{sumiilab}
% ベースになる色を指定できる
%\usecolortheme[named=Magenta]{structure}
% 数式の文字が細くて見難い時は serif の代わりに bold にしましょう
%\mathversion{bold}

%% ===============================================
%% スライドの表紙および PDF に表示される情報
%% ===============================================

%% 発表会の名前とか(省略可)
\session{TPP 2017}
%% スライドのタイトル
\title{Formal Verification of the Correspondence between Call-by-Need and Call-by-Name}
%% 必要ならば、サブタイトルも
%\subtitle{}
%% 発表者のお名前
\author{Masayuki Mizuno\\Joint work with Eijiro Sumii}
%% 発表者の所属([] 内は短い名前)
%\institute[東北大学 住井・松田研]{工学部 電気情報物理工学科\\住井・松田研究室}% 学部生
\institute[東北大学 住井・松田研]{Tohoku University}% 院生
%% 発表する日
\date{December 7, 2017}

%% ===============================================
%% 自動挿入される目次ページの設定(削除しても可)
%% ===============================================

%% section の先頭に自動挿入される目次ページ(削除すると、表示されなくなる)
\AtBeginSection[]{
\begin{frame}
  \frametitle{アウトライン}
  \tableofcontents[sectionstyle=show/shaded,subsectionstyle=show/show/hide]
\end{frame}}
%%% subsection の先頭に自動挿入される目次ページ(削除すると、表示されなくなる)
\AtBeginSubsection[]{
\begin{frame}
  \frametitle{アウトライン}
  \tableofcontents[sectionstyle=show/shaded,subsectionstyle=show/shaded/hide]
\end{frame}}
%
%% 現在の section 以外を非表示にする場合は以下のようにする

%% \AtBeginSection[]{
%% \begin{frame}
%%   \frametitle{アウトライン}
%%   \tableofcontents[sectionstyle=show/hide,subsectionstyle=show/show/hide]
%% \end{frame}}
%% \AtBeginSubsection[]{
%% \begin{frame}
%%   \frametitle{アウトライン}
%%   \tableofcontents[sectionstyle=show/hide,subsectionstyle=show/shaded/hide]
%% \end{frame}}

%% ===============================================
%% 定理環境の設定
%% ===============================================

%\setbeamertemplate{theorems}[numbered]% 定理環境に番号を付ける
\theoremstyle{definition}
\newtheorem{definition}{定義}
\newtheorem{axiom}{公理}
\newtheorem{theorem}{定理}
\newtheorem{lemma}{補題}
\newtheorem{corollary}{系}
\newtheorem{proposition}{命題}

%% ===============================================
%% ソースコードの設定
%% ===============================================

\usepackage{listings,jlisting}
%\usepackage[scale=0.9]{DejaVuSansMono}

\definecolor{DarkGreen}{rgb}{0,0.5,0}
% プログラミング言語と表示するフォント等の設定
\lstset{
  language={[Objective]Caml},% プログラミング言語
  basicstyle={\ttfamily\small},% ソースコードのテキストのスタイル
  keywordstyle={\bfseries},% 予約語等のキーワードのスタイル
  commentstyle={},% コメントのスタイル
  stringstyle={},% 文字列のスタイル
  frame=trlb,% ソースコードの枠線の設定 (none だと非表示)
  numbers=none,% 行番号の表示 (left だと左に表示)
  numberstyle={},% 行番号のスタイル
  xleftmargin=5pt,% 左余白
  xrightmargin=5pt,% 右余白
  keepspaces=true,% 空白を表示する
  mathescape,% $ で囲った部分を数式として表示する ($ がソースコード中で使えなくなるので注意)
  % 手動強調表示の設定
  moredelim=[is][\itshape]{@/}{/@},
  moredelim=[is][\color{red}]{@r\{}{\}@},
  moredelim=[is][\color{blue}]{@b\{}{\}@},
  moredelim=[is][\color{DarkGreen}]{@g\{}{\}@},
}

\newcommand{\LET}[3]{\mathbf{let}~#1=#2~\mathbf{in}~#3}
\newcommand{\EXPANDLET}[1]{#1^\pitchfork}
\newcommand{\SIZE}[2]{\parallel #1 \parallel_{#2}}
\newcommand{\FULLBETA}{\xrightarrow{\beta}}
\newcommand{\CALLBYNEED}{\xrightarrow{\mathrm{need}}}
\newcommand{\CALLBYNEEDI}{\xrightarrow{\mathrm{I}}}
\newcommand{\CALLBYNEEDVCA}{\xrightarrow{\mathrm{VCA}}}
\newcommand{\CALLBYNAME}{\xrightarrow{\mathrm{name}}}
\newcommand{\DEMAND}[2]{\mathbf{needs}(#2, #1)}
\newcommand{\STUCK}[2]{\mathbf{needs}_\mathrm{n}(#2, #1)}
\newcommand{\RTCLOS}[1]{#1_*}

%% ===============================================
%% 本文
%% ===============================================
\begin{document}
\frame[plain]{\titlepage}% タイトルページ

\begin{frame}
	\frametitle{動機: 非正格言語の仕様と実装のギャップ}
	\begin{itemize}
		\item 名前呼び: 非正格言語の(ハイレベルな)仕様
		\item 必要呼び: 非正格言語の実装
	\end{itemize}
	\vfill
	\begin{center}
		\Large
		これらの対応を検証したい
	\end{center}
\end{frame}

\begin{frame}
	\frametitle{$\lambda$計算の基本: full-$\beta$簡約}
	\begin{itemize}
		\item 簡約基は複数存在しうる
			\begin{itemize}
				\item $\underline{(\lambda xy.~y)~((\lambda x.~xx)~(\lambda x.~xx))}$
				\item $(\lambda xy.~y)~\underline{((\lambda x.~xx)~(\lambda x.~xx))}$
			\end{itemize}
		\item 下手に簡約基を選ぶと無限ループに
			\begin{itemize}
				\item 
					$\underline{(\lambda xy.~y)~((\lambda x.~xx)~(\lambda x.~xx))}$ \\
					$\FULLBETA \lambda y.~y$
				\item 
					$(\lambda xy.~y)~\underline{((\lambda x.~xx)~(\lambda x.~xx))}$ \\
					$\FULLBETA (\lambda xy.~y)~((\lambda x.~xx)~(\lambda x.~xx))$
			\end{itemize}
	\end{itemize}
\end{frame}

\begin{frame}
	\frametitle{非正格な言語の理論的背景: 名前呼び}
	\begin{definition}[名前呼び]
		\Large 関数呼び出し以前に引数を簡約せず,$\lambda$抽象の中も簡約しない評価戦略
	\end{definition}
	\begin{itemize}
		\item full-$\beta$簡約で$\lambda$抽象に評価できる項は名前呼びでも$\lambda$抽象まで評価できる\\
	\end{itemize}
	\large \[\underline{(\lambda xy.~x)~((\lambda x.~xx)~(\lambda x.~xx))} \RTCLOS{\CALLBYNAME} \lambda y.~y\]
\end{frame}

\begin{frame}
	\frametitle{名前呼びの問題点}
	\begin{itemize}
		\item 関数引数の簡約基が複製される
			\begin{itemize}
				\item 余計な簡約が必要になることも
			\end{itemize}
	\end{itemize}
	\vfill
	\begin{itemize}
		\item $(\lambda x.~xx)~\underline{((\lambda x.~x)~(\lambda x.~x))}$ \\
			$\FULLBETA (\lambda x.~xx)~(\lambda x.~x)$
		\item $\underline{(\lambda x.~xx)~((\lambda x.~x)~(\lambda x.~x))}$ \\
			$\CALLBYNAME (\lambda x.~x)~(\lambda x.~x)~((\lambda x.~x)~(\lambda x.~x))$
	\end{itemize}
\end{frame}

\begin{frame}
	\frametitle{名前呼びの改良: 必要呼び}
	\begin{itemize}
		\item 一度関数引数を評価したら値を覚えて使い回す
			\begin{itemize}
				\item 余計な簡約が不要に
			\end{itemize}
			\[\begin{array}{l}
				(\lambda x.~xx)~((\lambda x.~x)~(\lambda x.~x)) \\
				\CALLBYNEED \LET{x}{(\lambda x.~x)~(\lambda x.~x)}{x~x} \\
				\RTCLOS{\CALLBYNEED} \LET{x}{\lambda x.~x}{x~x}
			\end{array}\]
		\item 名前呼びと振る舞いが一致してほしい
	\end{itemize}
\end{frame}

\begin{frame}
	\frametitle{本研究の概要}
	\begin{itemize}
		\item 必要呼び$\lambda$計算[Ariola+ 1995]を検証に適するよう変形しCoqで定式化
			\begin{itemize}
				\item De Bruijnインデックス
				\item 評価文脈の排除
			\end{itemize}
		\item 名前呼びとの評価の対応に既存研究\\より簡単な証明を考案,Coqで定式化
			\begin{itemize}
				\item 標準化定理
			\end{itemize}
	\end{itemize}
\end{frame}

\section*{アウトライン}
\begin{frame}
  \frametitle{アウトライン}
  \tableofcontents[sectionstyle=show,subsectionstyle=hide]
\end{frame}

\section{対象言語: 名前呼び$\lambda$計算と必要呼び$\lambda$計算}

\begin{frame}
	\frametitle{名前呼び$\lambda$計算}
	\[ \begin{array}{llcl}
		\mbox{Terms}\quad & L,M,N\quad & ::=\quad & x~|~V~|~M~N \\
		\mbox{Values (WHNF)}\quad & V\quad & ::=\quad & \lambda x.M \\
		\mbox{Evaluation contexts}\quad & E_\mathrm{n}\quad & ::=\quad & [] ~|~ E_\mathrm{n}~M \\
	\end{array} \]
	\[ \begin{array}{lrcl}
		(\beta)\quad & (\lambda x.M)N & \rightarrow & M[x \mapsto N] \\
	\end{array} \]
	\begin{itemize}
		\item 簡約は決定的
		\item 行き詰まり状態は全て$E_\mathrm{n}[x]$の形
	\end{itemize}
\end{frame}

\begin{frame}
	\begin{lemma}[名前呼び$\lambda$計算の決定性]
	\begin{itemize}
		\item $\CALLBYNAME$は部分関数
		\item $E_\mathrm{n}[x]=E'_\mathrm{n}[y]$ならば$x=y$
		\item 任意の$M$について,以下のうちただ一つが成り立つ
			\begin{enumerate}
				\item $M$は値
				\item $M=E_\mathrm{n}[x]$となる$E_\mathrm{n}$と$x$が存在する
				\item $M$は$\CALLBYNAME$で簡約できる
			\end{enumerate}
	\end{itemize}
	\end{lemma}
\end{frame}

\begin{frame}
	\frametitle{名前呼びの性質: 標準化定理の2つの系}
	\large
	\begin{theorem}[正規化定理]
		$M$はfull-$\beta$簡約で$\lambda$抽象に評価できるならば,
		$M$は名前呼びの簡約でも$\lambda$抽象に評価できる
	\end{theorem}
	\begin{theorem}[準正規化定理]
		$M$はfull-$\beta$簡約で$\lambda$抽象に評価できるならば,
		${\CALLBYNAME}\circ{\RTCLOS{\FULLBETA}}$による$M$の簡約は必ず止まる
	\end{theorem}
	\begin{itemize}
		\item 必要呼びとの対応の証明に用いる
	\end{itemize}
\end{frame}

\begin{frame}
	\frametitle{必要呼び$\lambda$計算[Ariola+ 1995]}
	{\normalsize
		\[\begin{array}{llcl}
			\mbox{Terms} & M,N & ::= & x~|~V~|~M~N~|~\LET{x}{M}{N} \\
			\mbox{Values} & V & ::= & \lambda x.~M \\
			\mbox{Answers} & A & ::= & V~|~\LET{x}{M}{A} \\
			\mbox{Evaluation contexts} & E,E' & ::= & [] ~|~ E~M ~|~\LET{x}{M}{E} \\
			& & & ~|~ \LET{x}{E}{E'[x]} \\
		\end{array}\] }
	{\normalsize
		\[ \begin{array}{lrcl}
			(I) & (\lambda x.M)N & \rightarrow & \LET{x}{N}{M} \\
			(V) & \LET{x}{V}{E[x]} & \rightarrow & \LET{x}{V}{E[V]} \\
			(C) & (\LET{x}{M}{A})~N & \rightarrow & \LET{x}{M}{A~N} \\
			(A) & \LET{y}{(\LET{x}{M}{A})}{E[y]} & \rightarrow \\
			& \LET{x}{M}{\LET{y}{A}{E[y]}} & \\
		\end{array} \]}
	{\normalsize
		\[ \begin{array}{ll}
			\CALLBYNEEDI & \mbox{簡約規則(I)のみ用いる簡約}\\
			\CALLBYNEEDVCA & \mbox{簡約規則(V),(C)と(A)のみ用いる簡約}\\
		\end{array} \]}
\end{frame}

\begin{frame}
	\begin{lemma}[必要呼び$\lambda$計算の決定性]
	\begin{itemize}
		\item $\CALLBYNEEDI$は部分関数
		\item $\CALLBYNEEDVCA$は部分関数
		\item $E[x]=E'[y]$ならば$x=y$
		\item 任意の$M$について,以下のうちただ一つが成り立つ
			\begin{enumerate}
				\item $M$はanswer
				\item $M=E[x]$となる$E$と$x$が存在する
				\item $M$は$\CALLBYNEEDI$で簡約できる
				\item $M$は$\CALLBYNEEDVCA$で簡約できる
			\end{enumerate}
	\end{itemize}
	\end{lemma}
\end{frame}

\section{主定理の証明のアウトライン}

\begin{frame}
	\frametitle{本研究の主定理: 名前呼びと必要呼びの評価の対応}
	\large
	\begin{theorem}[必要呼び評価の健全性]
		$M$が必要呼びの簡約によってanswerに評価できるならば,
		$M$に対応する項$N$は名前呼びの簡約によって$\lambda$抽象に評価される
	\end{theorem}
	\begin{theorem}[必要呼び評価の完全性]
		$M$が名前呼びの簡約によって$\lambda$抽象に評価できるならば,
		$M$に対応する項$N$は必要呼びの簡約によってanswerに評価される
	\end{theorem}
\end{frame}

\begin{frame}[fragile]
	\frametitle{先行研究の証明方法}
	\begin{itemize}
		\item \verb|[Ariola+ 1995]|はAriolaらの研究とMaraistらの研究を合わせたもの
	\end{itemize}
	\begin{itembox}[c]{AriolaとFelleisenの証明[Ariola+ 1997]}
		\begin{itemize}
			\item Informalに定義されたグラフに基づく
		\end{itemize}
	\end{itembox}
	\begin{itembox}[c]{Maraistらの証明[Maraist+ 1998]}
		\begin{itemize}
			\item マーク付きの簡約と簡約位置を陽に扱うのが煩雑
			\item 本研究で一部参考
		\end{itemize}
	\end{itembox}
\end{frame}

\begin{frame}[fragile]
	\Large
	\begin{definition}[項同士の対応関係]
	\[ \begin{array}{rcl}
		\EXPANDLET{x} & = & x \\
		\EXPANDLET{(\lambda x. M)} & = & \lambda x. \EXPANDLET{M} \\
		\EXPANDLET{(M~N)} & = & \EXPANDLET{M}~\EXPANDLET{N} \\
		\EXPANDLET{(\LET{x}{M}{N})} & = & \EXPANDLET{N}[x \mapsto \EXPANDLET{M}] \\
	\end{array} \]
	\end{definition}
	\begin{itemize}
		\item $\mathbf{let}$を全て展開する関数
		\item マークを除くと[Maraist+ 1998]と一致
	\end{itemize}
\end{frame}

\begin{frame}
	\begin{lemma}[1ステップの対応]
	\begin{itemize}
		\item $\EXPANDLET{A}$は値
		\item 任意の$E$と$x$について,$\EXPANDLET{E[x]} = E_\mathrm{n}[x]$となる$E_\mathrm{n}$が存在
		\item $M \CALLBYNEEDVCA N$ならば$\EXPANDLET{M} = \EXPANDLET{N}$
		\item $M \CALLBYNEEDI N$ならば$\EXPANDLET{M}{\CALLBYNAME}\circ{\RTCLOS{\FULLBETA}} \EXPANDLET{N}$
	\end{itemize}
	\end{lemma}
\end{frame}

\begin{frame}
	\frametitle{再掲: 名前呼びと必要呼びの評価の対応}
	\Large
	\begin{theorem}[必要呼び評価の健全性]
		$M\RTCLOS{\CALLBYNEED}A$ならば,
		$\EXPANDLET{M}\RTCLOS{\CALLBYNAME}V$となるような値$V$が存在する
	\end{theorem}
	\begin{theorem}[必要呼び評価の完全性]
		$\EXPANDLET{M}\RTCLOS{\CALLBYNAME}V$ならば,
		$M\RTCLOS{\CALLBYNEED}A$となるようなanswer $A$が存在する
	\end{theorem}
	\begin{itemize}
		\item 評価結果$A$と$V$の間には$V\RTCLOS{\FULLBETA}\EXPANDLET{A}$が\\成り立つ(本発表では詳細は省きます)
	\end{itemize}
\end{frame}

\begin{frame}
	\frametitle{必要呼び評価の健全性の証明}
	\begin{proof}
		\Large
		$M\RTCLOS{\CALLBYNEED}A$と仮定すると,
		1ステップの対応により$\EXPANDLET{M}\RTCLOS{\FULLBETA}\EXPANDLET{A}$\\
		\medskip
		1ステップの対応により$\EXPANDLET{A}$は値であるから,正規化定理により$\EXPANDLET{M}\RTCLOS{\CALLBYNAME}V$となるような値$V$が存在する
	\end{proof}
\end{frame}

\begin{frame}
	\frametitle{完全性の証明の困難な点}
	\begin{itemize}
		\item Administrativeな簡約が止まらないかもしれない
			\begin{itemize}
				\item $M\CALLBYNEEDVCA N$ならば$\EXPANDLET{M}=\EXPANDLET{N}$
			\end{itemize}
		\item \textbf{let}で共有されている部分がすべて簡約される
			\begin{itemize}
				\item $M\CALLBYNEEDI N$ならば$\EXPANDLET{M}{\CALLBYNAME}\circ{\RTCLOS{\FULLBETA}}\EXPANDLET{N}$
				\item $\lambda$抽象の中も簡約されうる
			\end{itemize}
	\end{itemize}
\end{frame}

\begin{frame}
	\frametitle{$\CALLBYNEEDVCA$の正規化性}
	\Large
	\begin{proof}
		{[Maraist+ 1998]}の重み付けを変形したものにより証明
		{\normalsize \[ \begin{array}{rcl}
		\SIZE{x}{s} & = & s(x) \\
		\SIZE{\lambda x. M}{s} & = & \SIZE{M}{s \circ [x \mapsto 1]} \\
		\SIZE{M~N}{s} & = & 2\SIZE{M}{s}+2\SIZE{N}{s} \\
		\SIZE{\LET{x}{M}{N}}{s} & = & 2\SIZE{M}{s}+\SIZE{N}{s \circ [x \mapsto 1+\SIZE{M}{s}]}
		\end{array} \]}
	\end{proof}
\end{frame}

\begin{frame}
	\frametitle{必要呼びの完全性の証明(1/3)}
	\large
	\begin{proof}
		$\EXPANDLET{M}\RTCLOS{\CALLBYNAME}V$と仮定する.\\
		\smallskip
		まず,$\CALLBYNEED$によって$M$が正規化されることを示す.\\
		準正規化定理により,${\CALLBYNAME}\circ{\RTCLOS{\FULLBETA}}$による$\EXPANDLET{M}$の簡約は停止する.
		よって,$\CALLBYNEEDI$による簡約は${\CALLBYNAME}\circ{\RTCLOS{\FULLBETA}}$による簡約に対応している(1ステップの対応)ため,停止する.\\
		また,$\CALLBYNEEDVCA$の簡約も停止するため,
		$\CALLBYNEED$ ($= {\CALLBYNEEDI}\cup{\CALLBYNEEDVCA}$)による簡約は停止する.
		\phantom\qedhere
	\end{proof}
\end{frame}

\begin{frame}
	\frametitle{必要呼びの完全性の証明(2/3)}
	\large
	\begin{proof}
		次に,正規形$N$がanswerであることを示す.\\
		必要呼び$\lambda$計算の決定性により,$N$はanswerまたは行き詰まり状態$E[x]$である.\\
		$N$は行き詰まり状態$E[x]$と仮定すると,
		1ステップの対応により
		$\EXPANDLET{E[x]}=E_\mathrm{n}[x]$
		となるような$E_\mathrm{n}$が存在し,
		加えて$\EXPANDLET{M}\RTCLOS{\FULLBETA}\EXPANDLET{N}$が成り立つ.\phantom\qedhere
	\end{proof}
\end{frame}

\begin{frame}
	\frametitle{必要呼びの完全性の証明(3/3)}
	\large
	\begin{proof}
		ここで,full-$\beta$簡約の合流性により,
		$E_\mathrm{n}[x]\RTCLOS{\FULLBETA}L$かつ
		$A\RTCLOS{\FULLBETA}L$となるような$L$が存在する.
		名前呼び$\lambda$計算における値であることと行き詰まり状態であることはfull-$\beta$簡約により保存されるため,
		$L$は値かつ行き詰まり状態となってしまう.
		これは名前呼び$\lambda$計算の決定性に矛盾する.
	\end{proof}
\end{frame}

\section{Coqによる検証のポイント}

\subsection{束縛の表現}

\begin{frame}
	\frametitle{名前による束縛の表現の問題点}
	\begin{itemize}
		\item 項の等価性がCoqの等価性でなく\mbox{$\alpha$等価性}となる
			\[ \lambda x.\lambda y.x \overset{\alpha}{=} \lambda a.\lambda b.a \]
		\item Captureを避ける必要がある
			\[\begin{array}{l}
				[y \mapsto x](\lambda x.y) \\
				\overset{\alpha}{\cancel{=}}~\lambda x.x
			\end{array}\]
	\end{itemize}
\end{frame}

\begin{frame}
	\frametitle{De Bruijnインデックスによる束縛の表現}
	\begin{itemize}
		\item 内側から数えて何番目の束縛かで表現
			{\LARGE \[\begin{array}{lcl}
				\lambda x.x~(\lambda y.x~y) & \Rightarrow & \lambda.0~(\lambda.1~0)
			\end{array}\]}
		\item $\alpha$等価な式は文面上も同じ
			{\LARGE \[\begin{array}{lcl}
				\lambda x.\lambda y.x & \Rightarrow & \lambda.\lambda.1 \\
				\lambda a.\lambda b.a & \Rightarrow & \lambda.\lambda.1 \\
			\end{array}\]}
	\end{itemize}
\end{frame}

\subsection{評価文脈の除去}

\begin{frame}[fragile]
	\frametitle{評価文脈で困る例: 必要呼び$\lambda$計算の決定性}
\begin{lstlisting}
Lemma answer_or_stuck_or_reducible M :
  answer M
  \/ (exists E x,
    evalctx E /\ M = E.[tvar x] /\ bv E <= x)
  \/ (exists E L N,
    evalctx E /\ M = E.[L] /\ reduceI L N)
  \/ (exists E L N,
    evalctx E /\ M = E.[L] /\ reduceVCA L N).
\end{lstlisting}
	\begin{itemize}
		\item $M$についての帰納法で証明を試みる
	\end{itemize}
\end{frame}

\begin{frame}[fragile]
	\frametitle{必要呼び$\lambda$計算の決定性: $M=x$のケース}
\begin{lstlisting}
x : var
============================
answer M
\/ (exists E y,
  evalctx E /\ tvar x = E.[tvar y]
  /\ bv E <= y)
\/ (exists E L N,
  evalctx E /\ tvar x = E.[L] /\ reduceI L N)
\/ (exists E L N,
  evalctx E /\ tvar x = E.[L]
  /\ reduceVCA L N)
\end{lstlisting}
	\begin{itemize}
		\item $E$が$[]$のとき,明らかに成り立つ
	\end{itemize}
\end{frame}

\begin{frame}[fragile]
	\frametitle{必要呼び$\lambda$計算の決定性: $M=x$のケース}
	\begin{itemize}
		\item しかし,自動証明に失敗してしまう
	\end{itemize}
\begin{lstlisting}
Coq < eauto.
x : var
============================
answer M
\/ (exists E y,
  evalctx E /\ tvar x = E.[tvar y]
  /\ bv E <= y)
\/ (exists E L N,
  evalctx E /\ tvar x = E.[L] /\ reduceI L N)
\/ (exists E L N,
  evalctx E /\ tvar x = E.[L]
  /\ reduceVCA L N)
\end{lstlisting}
\end{frame}

\begin{frame}[fragile]
	\frametitle{なぜ自動証明に失敗するか?}
\Large
\begin{lstlisting}
exists E y, evalctx E /\
  tvar x = E.[tvar y] /\ bv E <= y
\end{lstlisting}
	を証明するためには,
\begin{lstlisting}
tvar x = E.[tvar y]
\end{lstlisting}
	となるような$E$を探さなくてはならない
	\vfill
	\begin{center}$\Downarrow$\end{center} % XXX 雑
\vfill
	\begin{center}
		高階単一化が必要
	\end{center}
\end{frame}

\begin{frame}
	\frametitle{解決策: 評価文脈の除去}
	\begin{itemize}
		\item 評価文脈のルールを簡約規則に展開
			\begin{itemize}
				\item $\FULLBETA$,$\CALLBYNAME$,$\CALLBYNEEDI$と$\CALLBYNEEDVCA$
			\end{itemize}
		\item 行き詰まり状態を表す述語の導入
			\begin{itemize}
				\item $\STUCK{x}{M}~(\iff \exists E. M=E_\mathrm{n}[x])$と$\DEMAND{x}{M}~(\iff \exists E. M=E[x])$
			\end{itemize}
		\item 評価規則\\$(V)~{\LET{x}{V}{E[x]}}~\rightarrow~{\LET{x}{V}{E[V]}}$
			を代入に変更
	\end{itemize}
\end{frame}

\begin{frame}[fragile]
	\frametitle{自動証明が回る例: 必要呼び$\lambda$計算の決定性}
\begin{lstlisting}
Lemma answer_or_stuck_or_reducible M :
  answer M \/
  (exists x, needs M x) \/
  (exists N, reduceI M N) \/
  (exists N, reduceVCA M N).
Proof.
  induction M as
    [|? [Hanswer|[[]|[[]|[]]]]
    ||? [Hanswer|[[]|[[]|[]]]]
      ? [|[[[]]|[[]|[]]]]]; eauto 6;
    inversion Hanswer; subst; eauto 6.
Qed.
\end{lstlisting}
\end{frame}

\begin{frame}
	\frametitle{完全性の証明の補足}
	\begin{itemize}
		\item ${\CALLBYNAME}\circ{\RTCLOS{\FULLBETA}}$による$\EXPANDLET{M}$の簡約が必ず停止することはaccessibility predicate \lstinline|Acc|で表現
		\item 必要呼び簡約の停止性は,\lstinline|Acc|についての帰納法の内側で$\CALLBYNEEDVCA$が停止することによる整礎帰納法を回して証明
	\end{itemize}
\end{frame}

\section{まとめ}

\begin{frame}
	\frametitle{まとめ}
	\begin{itemize}
		\item 必要呼び$\lambda$計算[Ariola+ 1995]を検証に適するよう変形しCoqで定式化
			\begin{itemize}
				\item De Bruijnインデックス
				\item 評価文脈の排除
			\end{itemize}
		\item 名前呼びとの評価の対応に既存研究\\より簡単な証明を考案,Coqで定式化
			\begin{itemize}
				\item 標準化定理
			\end{itemize}
	\end{itemize}
\end{frame}

\end{document}

